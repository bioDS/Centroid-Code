%% Generated by Sphinx.
\def\sphinxdocclass{report}
\documentclass[letterpaper,10pt,english]{sphinxmanual}
\ifdefined\pdfpxdimen
   \let\sphinxpxdimen\pdfpxdimen\else\newdimen\sphinxpxdimen
\fi \sphinxpxdimen=.75bp\relax

\PassOptionsToPackage{warn}{textcomp}
\usepackage[utf8]{inputenc}
\ifdefined\DeclareUnicodeCharacter
% support both utf8 and utf8x syntaxes
  \ifdefined\DeclareUnicodeCharacterAsOptional
    \def\sphinxDUC#1{\DeclareUnicodeCharacter{"#1}}
  \else
    \let\sphinxDUC\DeclareUnicodeCharacter
  \fi
  \sphinxDUC{00A0}{\nobreakspace}
  \sphinxDUC{2500}{\sphinxunichar{2500}}
  \sphinxDUC{2502}{\sphinxunichar{2502}}
  \sphinxDUC{2514}{\sphinxunichar{2514}}
  \sphinxDUC{251C}{\sphinxunichar{251C}}
  \sphinxDUC{2572}{\textbackslash}
\fi
\usepackage{cmap}
\usepackage[T1]{fontenc}
\usepackage{amsmath,amssymb,amstext}
\usepackage{babel}



\usepackage{times}
\expandafter\ifx\csname T@LGR\endcsname\relax
\else
% LGR was declared as font encoding
  \substitutefont{LGR}{\rmdefault}{cmr}
  \substitutefont{LGR}{\sfdefault}{cmss}
  \substitutefont{LGR}{\ttdefault}{cmtt}
\fi
\expandafter\ifx\csname T@X2\endcsname\relax
  \expandafter\ifx\csname T@T2A\endcsname\relax
  \else
  % T2A was declared as font encoding
    \substitutefont{T2A}{\rmdefault}{cmr}
    \substitutefont{T2A}{\sfdefault}{cmss}
    \substitutefont{T2A}{\ttdefault}{cmtt}
  \fi
\else
% X2 was declared as font encoding
  \substitutefont{X2}{\rmdefault}{cmr}
  \substitutefont{X2}{\sfdefault}{cmss}
  \substitutefont{X2}{\ttdefault}{cmtt}
\fi


\usepackage[Bjarne]{fncychap}
\usepackage{sphinx}

\fvset{fontsize=\small}
\usepackage{geometry}


% Include hyperref last.
\usepackage{hyperref}
% Fix anchor placement for figures with captions.
\usepackage{hypcap}% it must be loaded after hyperref.
% Set up styles of URL: it should be placed after hyperref.
\urlstyle{same}

\addto\captionsenglish{\renewcommand{\contentsname}{Contents:}}

\usepackage{sphinxmessages}
\setcounter{tocdepth}{1}



\title{treeoclock}
\date{Jun 22, 2021}
\release{0.0.1}
\author{Lars Berling, Lena Collienne, Jordan Kettles}
\newcommand{\sphinxlogo}{\vbox{}}
\renewcommand{\releasename}{Release}
\makeindex
\begin{document}

\pagestyle{empty}
\sphinxmaketitle
\pagestyle{plain}
\sphinxtableofcontents
\pagestyle{normal}
\phantomsection\label{\detokenize{index::doc}}



\chapter{Working with time trees}
\label{\detokenize{trees:working-with-time-trees}}\label{\detokenize{trees::doc}}
\begin{sphinxShadowBox}
\sphinxstyletopictitle{Contents}
\begin{itemize}
\item {} 
\sphinxAtStartPar
\phantomsection\label{\detokenize{trees:id1}}{\hyperref[\detokenize{trees:working-with-time-trees}]{\sphinxcrossref{Working with time trees}}}
\begin{itemize}
\item {} 
\sphinxAtStartPar
\phantomsection\label{\detokenize{trees:id2}}{\hyperref[\detokenize{trees:the-timetree-class}]{\sphinxcrossref{The TimeTree class}}}
\begin{itemize}
\item {} 
\sphinxAtStartPar
\phantomsection\label{\detokenize{trees:id3}}{\hyperref[\detokenize{trees:timetree-attributes}]{\sphinxcrossref{TimeTree attributes}}}

\item {} 
\sphinxAtStartPar
\phantomsection\label{\detokenize{trees:id4}}{\hyperref[\detokenize{trees:ete3-functionalities}]{\sphinxcrossref{ete3 functionalities}}}

\end{itemize}

\item {} 
\sphinxAtStartPar
\phantomsection\label{\detokenize{trees:id5}}{\hyperref[\detokenize{trees:the-timetreeset-class}]{\sphinxcrossref{The TimeTreeSet class}}}
\begin{itemize}
\item {} 
\sphinxAtStartPar
\phantomsection\label{\detokenize{trees:id6}}{\hyperref[\detokenize{trees:reading-trees}]{\sphinxcrossref{Reading Trees}}}

\item {} 
\sphinxAtStartPar
\phantomsection\label{\detokenize{trees:id7}}{\hyperref[\detokenize{trees:writing-trees}]{\sphinxcrossref{Writing trees}}}

\end{itemize}

\item {} 
\sphinxAtStartPar
\phantomsection\label{\detokenize{trees:id8}}{\hyperref[\detokenize{trees:random-trees}]{\sphinxcrossref{Random trees}}}

\item {} 
\sphinxAtStartPar
\phantomsection\label{\detokenize{trees:id9}}{\hyperref[\detokenize{trees:combining-multiple-timetreesets}]{\sphinxcrossref{Combining multiple TimeTreeSets}}}

\item {} 
\sphinxAtStartPar
\phantomsection\label{\detokenize{trees:id10}}{\hyperref[\detokenize{trees:general-functions}]{\sphinxcrossref{General Functions}}}

\item {} 
\sphinxAtStartPar
\phantomsection\label{\detokenize{trees:id11}}{\hyperref[\detokenize{trees:classes-for-the-c-library}]{\sphinxcrossref{Classes for the c library}}}
\begin{itemize}
\item {} 
\sphinxAtStartPar
\phantomsection\label{\detokenize{trees:id12}}{\hyperref[\detokenize{trees:node}]{\sphinxcrossref{NODE}}}

\item {} 
\sphinxAtStartPar
\phantomsection\label{\detokenize{trees:id13}}{\hyperref[\detokenize{trees:tree}]{\sphinxcrossref{TREE}}}

\item {} 
\sphinxAtStartPar
\phantomsection\label{\detokenize{trees:id14}}{\hyperref[\detokenize{trees:treelist}]{\sphinxcrossref{TREELIST}}}

\end{itemize}

\item {} 
\sphinxAtStartPar
\phantomsection\label{\detokenize{trees:id15}}{\hyperref[\detokenize{trees:class-converter-functions}]{\sphinxcrossref{Class converter functions}}}

\end{itemize}

\end{itemize}
\end{sphinxShadowBox}


\section{The TimeTree class}
\label{\detokenize{trees:the-timetree-class}}
\sphinxAtStartPar
A \sphinxcode{\sphinxupquote{TimeTree}} can be initialized with a given newick string using the \sphinxhref{http://etetoolkit.org/docs/latest/tutorial/tutorial\_trees.html\#trees}{ete3.Tree} constructor and its \sphinxhref{http://etetoolkit.org/docs/latest/tutorial/tutorial\_trees.html\#reading-and-writing-newick-trees}{format options}.
Additionally, a \sphinxcode{\sphinxupquote{TREE}} object ({\hyperref[\detokenize{trees:c-classes}]{\sphinxcrossref{\DUrole{std,std-ref}{Classes for the c library}}}}) is generated and saved in the \sphinxcode{\sphinxupquote{TimeTree}} and used for efficient distance computations.


\subsection{TimeTree attributes}
\label{\detokenize{trees:timetree-attributes}}

\begin{savenotes}\sphinxattablestart
\centering
\begin{tabulary}{\linewidth}[t]{|T|T|}
\hline
\sphinxstyletheadfamily 
\sphinxAtStartPar
Method
&\sphinxstyletheadfamily 
\sphinxAtStartPar
Description
\\
\hline
\sphinxAtStartPar
\sphinxcode{\sphinxupquote{TimeTree.etree}}
&
\sphinxAtStartPar
returns the \sphinxcode{\sphinxupquote{ete3.Tree}} object
\\
\hline
\sphinxAtStartPar
\sphinxcode{\sphinxupquote{TimeTree.ctree}}
&
\sphinxAtStartPar
returns the respective \sphinxcode{\sphinxupquote{TREE}} object
\\
\hline
\sphinxAtStartPar
\sphinxcode{\sphinxupquote{len(TimeTree)}}
&
\sphinxAtStartPar
returns the number of leaves of the \sphinxcode{\sphinxupquote{TimeTree}}
\\
\hline
\sphinxAtStartPar
\sphinxcode{\sphinxupquote{TimeTree.fp\_distance(t)}}
&
\sphinxAtStartPar
returns the findpath distance to another \sphinxcode{\sphinxupquote{TimeTree}} \sphinxstyleemphasis{t}
\\
\hline
\sphinxAtStartPar
\sphinxcode{\sphinxupquote{TimeTree.fp\_path(t)}}
&
\sphinxAtStartPar
returns a \sphinxstyleemphasis{list} of \sphinxcode{\sphinxupquote{TREE}} objects
\\
\hline
\sphinxAtStartPar
\sphinxcode{\sphinxupquote{TimeTree.get\_newick(format)}}
&
\sphinxAtStartPar
returns the \sphinxstyleemphasis{write()} function of the \sphinxcode{\sphinxupquote{ete3.Tree}} with the specified \sphinxstyleemphasis{format}, defaults to \sphinxstyleemphasis{format=5}
\\
\hline
\sphinxAtStartPar
\sphinxcode{\sphinxupquote{TimeTree.copy()}}
&
\sphinxAtStartPar
returns a deep copy of the current \sphinxcode{\sphinxupquote{TimeTree}} (specifically used to generate a new \sphinxcode{\sphinxupquote{TREE}} object)
\\
\hline
\sphinxAtStartPar
\sphinxcode{\sphinxupquote{TimeTree.neighbours()}}
&
\sphinxAtStartPar
returns a list of \sphinxcode{\sphinxupquote{TimeTree}}’s containing all neighbours at distance \sphinxstyleemphasis{1}
\\
\hline
\sphinxAtStartPar
\sphinxcode{\sphinxupquote{TimeTree.rank\_neighbours()}}
&
\sphinxAtStartPar
returns a list of \sphinxcode{\sphinxupquote{TimeTree}}’s containing only neighbours one \sphinxstyleemphasis{rank move} away
\\
\hline
\sphinxAtStartPar
\sphinxcode{\sphinxupquote{TimeTree.nni\_neighbours()}}
&
\sphinxAtStartPar
returns a list of \sphinxcode{\sphinxupquote{TimeTree}}’s containing only neighbours one \sphinxstyleemphasis{NNI move} away
\\
\hline
\end{tabulary}
\par
\sphinxattableend\end{savenotes}

\sphinxAtStartPar
This is an example of how to access the different attributes of a TimeTree object:

\begin{sphinxVerbatim}[commandchars=\\\{\}]
\PYG{k+kn}{from} \PYG{n+nn}{treeoclock}\PYG{n+nn}{.}\PYG{n+nn}{trees}\PYG{n+nn}{.}\PYG{n+nn}{time\PYGZus{}trees} \PYG{k+kn}{import} \PYG{n}{TimeTree}


\PYG{c+c1}{\PYGZsh{} Initialize a time tree from a newick string}
\PYG{n}{tt} \PYG{o}{=} \PYG{n}{TimeTree}\PYG{p}{(}\PYG{l+s+s2}{\PYGZdq{}}\PYG{l+s+s2}{((1:3,5:3):1,(4:2,(3:1,2:1):1):2);}\PYG{l+s+s2}{\PYGZdq{}}\PYG{p}{)}

\PYG{n}{tt}\PYG{o}{.}\PYG{n}{ctree}  \PYG{c+c1}{\PYGZsh{} the TREE class object}

\PYG{n}{tt}\PYG{o}{.}\PYG{n}{etree}  \PYG{c+c1}{\PYGZsh{} the ete3.Tree object}

\PYG{n+nb}{len}\PYG{p}{(}\PYG{n}{tt}\PYG{p}{)}  \PYG{c+c1}{\PYGZsh{} Number of leaves in the tree tt \PYGZhy{}\PYGZhy{}\PYGZgt{} 5}

\PYG{n}{tt}\PYG{o}{.}\PYG{n}{fp\PYGZus{}distance}\PYG{p}{(}\PYG{n}{tt}\PYG{p}{)}  \PYG{c+c1}{\PYGZsh{} Distance to another TimeTree \PYGZhy{}\PYGZhy{}\PYGZgt{} 0}

\PYG{n}{tt}\PYG{o}{.}\PYG{n}{fp\PYGZus{}path}\PYG{p}{(}\PYG{n}{tt}\PYG{p}{)}  \PYG{c+c1}{\PYGZsh{} A shortest path to another TimeTree \PYGZhy{}\PYGZhy{}\PYGZgt{} []}

\PYG{n}{tt}\PYG{o}{.}\PYG{n}{get\PYGZus{}newick}\PYG{p}{(}\PYG{p}{)}  \PYG{c+c1}{\PYGZsh{} Returns the newick string in ete3 format=5}

\PYG{n}{ttc} \PYG{o}{=} \PYG{n}{tt}\PYG{o}{.}\PYG{n}{copy}\PYG{p}{(}\PYG{p}{)}  \PYG{c+c1}{\PYGZsh{} ttc contains a deep copy of the TimeTree tt}

\PYG{n}{tt}\PYG{o}{.}\PYG{n}{neighbours}\PYG{p}{(}\PYG{p}{)}  \PYG{c+c1}{\PYGZsh{} a list of TimeTree objects each at distance one to tt}

\PYG{n}{tt}\PYG{o}{.}\PYG{n}{rank\PYGZus{}neighbours}\PYG{p}{(}\PYG{p}{)}  \PYG{c+c1}{\PYGZsh{} list of TimeTree obtained by doing all possible rank moves on tt}

\PYG{n}{tt}\PYG{o}{.}\PYG{n}{nni\PYGZus{}neighbours}\PYG{p}{(}\PYG{p}{)}  \PYG{c+c1}{\PYGZsh{} list of TimeTree obtained by doing all possible NNI moves on tt}
\end{sphinxVerbatim}


\subsection{ete3 functionalities}
\label{\detokenize{trees:ete3-functionalities}}
\sphinxAtStartPar
Via the \sphinxcode{\sphinxupquote{ete3.Tree}} object the respective function of the \sphinxcode{\sphinxupquote{ete3}} package are available for a \sphinxcode{\sphinxupquote{TimeTree}} object.
For example drawing and saving a tree to a file:

\begin{sphinxVerbatim}[commandchars=\\\{\}]
\PYG{k+kn}{from} \PYG{n+nn}{treeoclock}\PYG{n+nn}{.}\PYG{n+nn}{trees}\PYG{n+nn}{.}\PYG{n+nn}{time\PYGZus{}trees} \PYG{k+kn}{import} \PYG{n}{TimeTree}

\PYG{n}{tt} \PYG{o}{=} \PYG{n}{TimeTree}\PYG{p}{(}\PYG{l+s+s2}{\PYGZdq{}}\PYG{l+s+s2}{((1:3,5:3):1,(4:2,(3:1,2:1):1):2);}\PYG{l+s+s2}{\PYGZdq{}}\PYG{p}{)}

\PYG{c+c1}{\PYGZsh{} Automatically save the tree to a specific file\PYGZus{}path location}
\PYG{n}{tt}\PYG{o}{.}\PYG{n}{etree}\PYG{o}{.}\PYG{n}{render}\PYG{p}{(}\PYG{l+s+s1}{\PYGZsq{}}\PYG{l+s+s1}{file\PYGZus{}path\PYGZus{}string}\PYG{l+s+s1}{\PYGZsq{}}\PYG{p}{)}

\PYG{c+c1}{\PYGZsh{} Defining a layout to display internal node names in the plot}
\PYG{k}{def} \PYG{n+nf}{my\PYGZus{}layout}\PYG{p}{(}\PYG{n}{node}\PYG{p}{)}\PYG{p}{:}
    \PYG{k}{if} \PYG{n}{node}\PYG{o}{.}\PYG{n}{is\PYGZus{}leaf}\PYG{p}{(}\PYG{p}{)}\PYG{p}{:}
        \PYG{c+c1}{\PYGZsh{} If terminal node, draws its name}
        \PYG{n}{name\PYGZus{}face} \PYG{o}{=} \PYG{n}{ete3}\PYG{o}{.}\PYG{n}{AttrFace}\PYG{p}{(}\PYG{l+s+s2}{\PYGZdq{}}\PYG{l+s+s2}{name}\PYG{l+s+s2}{\PYGZdq{}}\PYG{p}{)}
    \PYG{k}{else}\PYG{p}{:}
        \PYG{c+c1}{\PYGZsh{} If internal node, draws label with smaller font size}
        \PYG{n}{name\PYGZus{}face} \PYG{o}{=} \PYG{n}{ete3}\PYG{o}{.}\PYG{n}{AttrFace}\PYG{p}{(}\PYG{l+s+s2}{\PYGZdq{}}\PYG{l+s+s2}{name}\PYG{l+s+s2}{\PYGZdq{}}\PYG{p}{,} \PYG{n}{fsize}\PYG{o}{=}\PYG{l+m+mi}{10}\PYG{p}{)}
    \PYG{c+c1}{\PYGZsh{} Adds the name face to the image at the preferred position}
    \PYG{n}{ete3}\PYG{o}{.}\PYG{n}{faces}\PYG{o}{.}\PYG{n}{add\PYGZus{}face\PYGZus{}to\PYGZus{}node}\PYG{p}{(}\PYG{n}{name\PYGZus{}face}\PYG{p}{,} \PYG{n}{node}\PYG{p}{,} \PYG{n}{column}\PYG{o}{=}\PYG{l+m+mi}{0}\PYG{p}{,} \PYG{n}{position}\PYG{o}{=}\PYG{l+s+s2}{\PYGZdq{}}\PYG{l+s+s2}{branch\PYGZhy{}right}\PYG{l+s+s2}{\PYGZdq{}}\PYG{p}{)}

\PYG{n}{ts} \PYG{o}{=} \PYG{n}{ete3}\PYG{o}{.}\PYG{n}{TreeStyle}\PYG{p}{(}\PYG{p}{)}
\PYG{n}{ts}\PYG{o}{.}\PYG{n}{show\PYGZus{}leaf\PYGZus{}name} \PYG{o}{=} \PYG{k+kc}{False}
\PYG{n}{ts}\PYG{o}{.}\PYG{n}{layout\PYGZus{}fn} \PYG{o}{=} \PYG{n}{my\PYGZus{}layout}
\PYG{n}{ts}\PYG{o}{.}\PYG{n}{show\PYGZus{}branch\PYGZus{}length} \PYG{o}{=} \PYG{k+kc}{True}
\PYG{n}{ts}\PYG{o}{.}\PYG{n}{show\PYGZus{}scale} \PYG{o}{=} \PYG{k+kc}{False}

\PYG{c+c1}{\PYGZsh{} Will open a separate plot window, which also allows interactive changes and saving the image}
\PYG{n}{tt}\PYG{o}{.}\PYG{n}{etree}\PYG{o}{.}\PYG{n}{show}\PYG{p}{(}\PYG{n}{tree\PYGZus{}style}\PYG{o}{=}\PYG{n}{ts}\PYG{p}{)}
\end{sphinxVerbatim}

\sphinxAtStartPar
See the \sphinxcode{\sphinxupquote{ete3}} \sphinxhref{http://etetoolkit.org/docs/latest/tutorial/tutorial\_drawing.html}{documentation} for more options.


\section{The TimeTreeSet class}
\label{\detokenize{trees:the-timetreeset-class}}
\sphinxAtStartPar
A \sphinxcode{\sphinxupquote{TimeTreeSet}} is an iterable list of \sphinxcode{\sphinxupquote{TimeTree}} objects, which can be initialized with a nexus file.


\begin{savenotes}\sphinxattablestart
\centering
\begin{tabulary}{\linewidth}[t]{|T|T|}
\hline
\sphinxstyletheadfamily 
\sphinxAtStartPar
Method
&\sphinxstyletheadfamily 
\sphinxAtStartPar
Description
\\
\hline
\sphinxAtStartPar
\sphinxcode{\sphinxupquote{TimeTreeSet.map}}
&
\sphinxAtStartPar
a dictionary conataining the taxa to integer translation from the nexus file
\\
\hline
\sphinxAtStartPar
\sphinxcode{\sphinxupquote{TimeTreeSet.trees}}
&
\sphinxAtStartPar
a list of \sphinxcode{\sphinxupquote{TimeTree}} objects
\\
\hline
\sphinxAtStartPar
\sphinxcode{\sphinxupquote{TimeTreeSet{[}i{]}}}
&
\sphinxAtStartPar
returns the \sphinxcode{\sphinxupquote{TimeTree}} at \sphinxcode{\sphinxupquote{TimeTreeSet.trees{[}i{]}}}
\\
\hline
\sphinxAtStartPar
\sphinxcode{\sphinxupquote{len(TimeTreeSet)}}
&
\sphinxAtStartPar
returns the number of trees in the list \sphinxcode{\sphinxupquote{TimeTreeSet.trees}}
\\
\hline
\sphinxAtStartPar
\sphinxcode{\sphinxupquote{TimeTreeSet.fp\_distance(i, j)}}
&
\sphinxAtStartPar
returns the distances between the trees at postition i and j
\\
\hline
\sphinxAtStartPar
\sphinxcode{\sphinxupquote{TimeTreeSet.fp\_path(i, j)}}
&
\sphinxAtStartPar
returns a shortest path (list of \sphinxcode{\sphinxupquote{TREE}}) between the trees at postition i and j
\\
\hline
\end{tabulary}
\par
\sphinxattableend\end{savenotes}


\subsection{Reading Trees}
\label{\detokenize{trees:reading-trees}}
\sphinxAtStartPar
A TimeTreeSet object can be initialized with a path to a nexus file.

\begin{sphinxVerbatim}[commandchars=\\\{\}]
\PYG{k+kn}{from} \PYG{n+nn}{treeoclock}\PYG{n+nn}{.}\PYG{n+nn}{trees}\PYG{n+nn}{.}\PYG{n+nn}{time\PYGZus{}trees} \PYG{k+kn}{import} \PYG{n}{TimeTreeSet}


\PYG{c+c1}{\PYGZsh{} Initializing with a path to a nexus tree file}
\PYG{n}{tts} \PYG{o}{=} \PYG{n}{TimeTreeSet}\PYG{p}{(}\PYG{l+s+s2}{\PYGZdq{}}\PYG{l+s+s2}{path\PYGZus{}to\PYGZus{}nexus\PYGZus{}file.nex}\PYG{l+s+s2}{\PYGZdq{}}\PYG{p}{)}

\PYG{n}{tts}\PYG{o}{.}\PYG{n}{map}  \PYG{c+c1}{\PYGZsh{} a dictionary keys:int and values:string(taxa)}

\PYG{n}{tts}\PYG{o}{.}\PYG{n}{trees}  \PYG{c+c1}{\PYGZsh{} A list of TimeTree objects}

\PYG{k}{for} \PYG{n}{tree} \PYG{o+ow}{in} \PYG{n}{tts}\PYG{p}{:}
    \PYG{c+c1}{\PYGZsh{} tree is a TimeTree object}
    \PYG{o}{.}\PYG{o}{.}\PYG{o}{.}
\PYG{n}{tts}\PYG{p}{[}\PYG{l+m+mi}{0}\PYG{p}{]}  \PYG{c+c1}{\PYGZsh{} trees are accessible via the index}

\PYG{n+nb}{len}\PYG{p}{(}\PYG{n}{tts}\PYG{p}{)}  \PYG{c+c1}{\PYGZsh{} Returns the number of trees in the TimeTreeSet object}

\PYG{n}{tts}\PYG{o}{.}\PYG{n}{fp\PYGZus{}distance}\PYG{p}{(}\PYG{n}{i}\PYG{p}{,} \PYG{n}{j}\PYG{p}{)}  \PYG{c+c1}{\PYGZsh{} Returns the distance between trees i and j}
\PYG{n}{tts}\PYG{o}{.}\PYG{n}{fp\PYGZus{}path}\PYG{p}{(}\PYG{n}{i}\PYG{p}{,} \PYG{n}{j}\PYG{p}{)}  \PYG{c+c1}{\PYGZsh{} Returns a shortest path between trees i and j}
\end{sphinxVerbatim}


\subsection{Writing trees}
\label{\detokenize{trees:writing-trees}}
\sphinxAtStartPar
Still WIP


\section{Random trees}
\label{\detokenize{trees:random-trees}}
\sphinxAtStartPar
STILL WIP


\section{Combining multiple TimeTreeSets}
\label{\detokenize{trees:combining-multiple-timetreesets}}
\sphinxAtStartPar
Still WIP


\section{General Functions}
\label{\detokenize{trees:general-functions}}
\sphinxAtStartPar
A list of the direct functions and their arguments.


\begin{savenotes}\sphinxattablestart
\centering
\begin{tabulary}{\linewidth}[t]{|T|T|}
\hline
\sphinxstyletheadfamily 
\sphinxAtStartPar
Function
&\sphinxstyletheadfamily 
\sphinxAtStartPar
Description
\\
\hline
\sphinxAtStartPar
\sphinxcode{\sphinxupquote{time\_trees.neighbourhood(tree)}}
&
\sphinxAtStartPar
returns a list of \sphinxcode{\sphinxupquote{TimeTree}} objects containing the one\sphinxhyphen{}neighbours of tree
\\
\hline
\sphinxAtStartPar
\sphinxcode{\sphinxupquote{time\_trees.get\_rank\_neighbours(tree)}}
&
\sphinxAtStartPar
returns a list of \sphinxcode{\sphinxupquote{TimeTree}} objects containing the rank neighbours of tree
\\
\hline
\sphinxAtStartPar
\sphinxcode{\sphinxupquote{time\_trees.get\_nni\_neighbours(tree)}}
&
\sphinxAtStartPar
returns a list of \sphinxcode{\sphinxupquote{TimeTree}} objects containing the NNI neighbours of tree
\\
\hline
\sphinxAtStartPar
\sphinxcode{\sphinxupquote{time\_trees.read\_nexus(file)}}
&
\sphinxAtStartPar
returns a list of \sphinxcode{\sphinxupquote{TimeTree}} objects contained in given the nexus file
\\
\hline
\sphinxAtStartPar
\sphinxcode{\sphinxupquote{time\_trees.get\_mapping\_dict(file)}}
&
\sphinxAtStartPar
returns a dictionary containg the taxa to integer transaltion of the given file
\\
\hline
\sphinxAtStartPar
\sphinxcode{\sphinxupquote{time\_trees.findpath\_distance(t1, t2)}}
&
\sphinxAtStartPar
Computes the distance between t1 and t2
\\
\hline
\sphinxAtStartPar
\sphinxcode{\sphinxupquote{time\_trees.findpath\_path(t1, t2)}}
&
\sphinxAtStartPar
Computes the path between t1 and t2
\\
\hline
\end{tabulary}
\par
\sphinxattableend\end{savenotes}

\begin{sphinxadmonition}{note}{Note:}
\sphinxAtStartPar
Both functions \sphinxcode{\sphinxupquote{time\_trees.findpath\_distance(t1, t2)}} and \sphinxcode{\sphinxupquote{time\_trees.findpath\_path(t1, t2)}}
can be called with t1 and t2 being either a \sphinxcode{\sphinxupquote{TREE}}, \sphinxcode{\sphinxupquote{TimeTree}} or \sphinxcode{\sphinxupquote{ete3.Tree}}
\end{sphinxadmonition}


\section{Classes for the c library}
\label{\detokenize{trees:classes-for-the-c-library}}\label{\detokenize{trees:c-classes}}
\sphinxAtStartPar
These classes are found in the \sphinxcode{\sphinxupquote{\_ctrees.py}} module.
The corresponding CDLL c library is generated from \sphinxcode{\sphinxupquote{findpath.c}}.


\subsection{NODE}
\label{\detokenize{trees:node}}\begin{itemize}
\item {} 
\sphinxAtStartPar
\sphinxcode{\sphinxupquote{parent}}: index of the parent node (int, defaults to \sphinxhyphen{}1)

\item {} 
\sphinxAtStartPar
\sphinxcode{\sphinxupquote{children{[}2{]}}}: index of the two children ({[}int{]}, defaults to {[}\sphinxhyphen{}1, \sphinxhyphen{}1{]})

\item {} 
\sphinxAtStartPar
\sphinxcode{\sphinxupquote{time}}: Time of the node (int, defaults to 0)

\end{itemize}

\begin{sphinxadmonition}{note}{Note:}
\sphinxAtStartPar
The attribute \sphinxcode{\sphinxupquote{time}} is currently not being used!
\end{sphinxadmonition}


\subsection{TREE}
\label{\detokenize{trees:tree}}\begin{itemize}
\item {} 
\sphinxAtStartPar
\sphinxcode{\sphinxupquote{num\_leaves}}: Number of leaves in the tree (int)

\item {} 
\sphinxAtStartPar
\sphinxcode{\sphinxupquote{tree}}: Points to a \sphinxcode{\sphinxupquote{NODE}} object (POINTER(\sphinxcode{\sphinxupquote{NODE}}))

\item {} 
\sphinxAtStartPar
\sphinxcode{\sphinxupquote{root\_time}}: Time of the root \sphinxcode{\sphinxupquote{Node}} (int)

\end{itemize}

\begin{sphinxadmonition}{note}{Note:}
\sphinxAtStartPar
The attribute \sphinxcode{\sphinxupquote{root\_time}} is currently not being used!
\end{sphinxadmonition}


\subsection{TREELIST}
\label{\detokenize{trees:treelist}}\begin{itemize}
\item {} 
\sphinxAtStartPar
\sphinxcode{\sphinxupquote{num\_trees}}: Number of trees in the list (int)

\item {} 
\sphinxAtStartPar
\sphinxcode{\sphinxupquote{trees}}: List of trees (POINTER(\sphinxcode{\sphinxupquote{TREE}}))

\end{itemize}


\section{Class converter functions}
\label{\detokenize{trees:class-converter-functions}}
\sphinxAtStartPar
These are found in \sphinxcode{\sphinxupquote{\_converter.py}}.


\begin{savenotes}\sphinxattablestart
\centering
\begin{tabulary}{\linewidth}[t]{|T|T|}
\hline
\sphinxstyletheadfamily 
\sphinxAtStartPar
Function
&\sphinxstyletheadfamily 
\sphinxAtStartPar
Description
\\
\hline
\sphinxAtStartPar
\sphinxcode{\sphinxupquote{\_converter.ete3\_to\_ctree(tree)}}
&
\sphinxAtStartPar
traverses an \sphinxcode{\sphinxupquote{ete3.Tree}} and construct the correct \sphinxcode{\sphinxupquote{TREE}}
\\
\hline
\sphinxAtStartPar
\sphinxcode{\sphinxupquote{\_converter.ctree\_to\_ete3(ctree)}}
&
\sphinxAtStartPar
recursively traverses a \sphinxcode{\sphinxupquote{TREE}} and generates an \sphinxcode{\sphinxupquote{ete3.Tree}}
\\
\hline
\end{tabulary}
\par
\sphinxattableend\end{savenotes}


\chapter{Summarizing trees}
\label{\detokenize{summary:summarizing-trees}}\label{\detokenize{summary::doc}}
\begin{sphinxShadowBox}
\sphinxstyletopictitle{Contents}
\begin{itemize}
\item {} 
\sphinxAtStartPar
\phantomsection\label{\detokenize{summary:id1}}{\hyperref[\detokenize{summary:summarizing-trees}]{\sphinxcrossref{Summarizing trees}}}
\begin{itemize}
\item {} 
\sphinxAtStartPar
\phantomsection\label{\detokenize{summary:id2}}{\hyperref[\detokenize{summary:the-centroid-class}]{\sphinxcrossref{The Centroid class}}}
\begin{itemize}
\item {} 
\sphinxAtStartPar
\phantomsection\label{\detokenize{summary:id3}}{\hyperref[\detokenize{summary:centroid-attributes}]{\sphinxcrossref{Centroid attributes}}}

\item {} 
\sphinxAtStartPar
\phantomsection\label{\detokenize{summary:id4}}{\hyperref[\detokenize{summary:centroid-variations}]{\sphinxcrossref{Centroid variations}}}

\item {} 
\sphinxAtStartPar
\phantomsection\label{\detokenize{summary:id5}}{\hyperref[\detokenize{summary:computing-the-sos}]{\sphinxcrossref{Computing the SoS}}}

\item {} 
\sphinxAtStartPar
\phantomsection\label{\detokenize{summary:id6}}{\hyperref[\detokenize{summary:sampling-the-neighbourhood}]{\sphinxcrossref{Sampling the neighbourhood}}}

\end{itemize}

\item {} 
\sphinxAtStartPar
\phantomsection\label{\detokenize{summary:id7}}{\hyperref[\detokenize{summary:frechet-mean}]{\sphinxcrossref{Frechet Mean}}}

\end{itemize}

\end{itemize}
\end{sphinxShadowBox}


\section{The Centroid class}
\label{\detokenize{summary:the-centroid-class}}
\sphinxAtStartPar
TODO: Change the start default to the Frechet Mean algorithm instead of last

\begin{sphinxVerbatim}[commandchars=\\\{\}]
\PYG{k}{class} \PYG{n+nc}{treeoclock}\PYG{o}{.}\PYG{n}{summary}\PYG{o}{.}\PYG{n}{Centroid}\PYG{p}{(}\PYG{n}{variation}\PYG{o}{=}\PYG{l+s+s2}{\PYGZdq{}}\PYG{l+s+s2}{greedy}\PYG{l+s+s2}{\PYGZdq{}}\PYG{p}{,} \PYG{n}{n\PYGZus{}cores}\PYG{o}{=}\PYG{k+kc}{None}\PYG{p}{,}
                                  \PYG{n}{select}\PYG{o}{=}\PYG{l+s+s1}{\PYGZsq{}}\PYG{l+s+s1}{random}\PYG{l+s+s1}{\PYGZsq{}}\PYG{p}{,} \PYG{n}{start}\PYG{o}{=}\PYG{l+s+s1}{\PYGZsq{}}\PYG{l+s+s1}{last}\PYG{l+s+s1}{\PYGZsq{}}\PYG{p}{)}
\end{sphinxVerbatim}

\sphinxAtStartPar
Applying a variation of the centroid algorithm on a set of trees.


\subsection{Centroid attributes}
\label{\detokenize{summary:centroid-attributes}}

\begin{savenotes}\sphinxattablestart
\centering
\begin{tabulary}{\linewidth}[t]{|T|T|}
\hline
\sphinxstyletheadfamily 
\sphinxAtStartPar
Method
&\sphinxstyletheadfamily 
\sphinxAtStartPar
Description
\\
\hline
\sphinxAtStartPar
\sphinxcode{\sphinxupquote{Centroid.variation}}
&
\sphinxAtStartPar
Accesses the variation parameter which has to be in the list of specified variations.
\\
\hline
\sphinxAtStartPar
\sphinxcode{\sphinxupquote{Centroid.n\_cores}}
&
\sphinxAtStartPar
Accesses the number of Threads that will be used to compute the SoS value, using ThreadPool.
\\
\hline
\sphinxAtStartPar
\sphinxcode{\sphinxupquote{Centroid.select}}
&
\sphinxAtStartPar
Accesses the selection parameter, in case of multiple options choose a selected tree i.e. the first, last or a random tree.
\\
\hline
\sphinxAtStartPar
\sphinxcode{\sphinxupquote{Centroid.start}}
&
\sphinxAtStartPar
Accesses the start parameter, specifying at which point to start the centroid computation.
\\
\hline
\sphinxAtStartPar
\sphinxcode{\sphinxupquote{Centroid.compute\_centroid(trees)}}
&
\sphinxAtStartPar
Computes the centroid for a given set of trees, using the options that were selected via the other attributes of the \sphinxcode{\sphinxupquote{Centroid}}.
\\
\hline
\end{tabulary}
\par
\sphinxattableend\end{savenotes}

\sphinxAtStartPar
Examples of how to set and use the attributes of a \sphinxcode{\sphinxupquote{Centroid}}:

\begin{sphinxVerbatim}[commandchars=\\\{\}]
\PYG{k+kn}{from} \PYG{n+nn}{treeoclock}\PYG{n+nn}{.}\PYG{n+nn}{trees}\PYG{n+nn}{.}\PYG{n+nn}{time\PYGZus{}trees} \PYG{k+kn}{import} \PYG{n}{TimeTreeSet}
\PYG{k+kn}{from} \PYG{n+nn}{treeoclock}\PYG{n+nn}{.}\PYG{n+nn}{summary}\PYG{n+nn}{.}\PYG{n+nn}{centroid} \PYG{k+kn}{import} \PYG{n}{Centroid}


\PYG{c+c1}{\PYGZsh{} Initializing an empty Centroid class}
\PYG{n}{mycen} \PYG{o}{=} \PYG{n}{Centroid}\PYG{p}{(}\PYG{p}{)}

\PYG{n}{mycen}\PYG{o}{.}\PYG{n}{variation} \PYG{o}{=} \PYG{l+s+s2}{\PYGZdq{}}\PYG{l+s+s2}{inc\PYGZus{}sub}\PYG{l+s+s2}{\PYGZdq{}}  \PYG{c+c1}{\PYGZsh{} Changing the variation}

\PYG{n}{mycen}\PYG{o}{.}\PYG{n}{n\PYGZus{}cores} \PYG{o}{=} \PYG{l+m+mi}{4}  \PYG{c+c1}{\PYGZsh{} Setting the number of Threads to use}

\PYG{n}{mycen}\PYG{o}{.}\PYG{n}{select} \PYG{o}{=} \PYG{l+s+s2}{\PYGZdq{}}\PYG{l+s+s2}{first}\PYG{l+s+s2}{\PYGZdq{}}  \PYG{c+c1}{\PYGZsh{} Selection parameter set to the first tree found}

\PYG{n}{mycen}\PYG{o}{.}\PYG{n}{start} \PYG{o}{=} \PYG{l+m+mi}{6}  \PYG{c+c1}{\PYGZsh{} Starting the algorithm from the sixth tree in the set (Error if non\PYGZhy{}existant)}

\PYG{c+c1}{\PYGZsh{} Computing the centroid for an empty TimeTreeSet}
\PYG{n}{cen\PYGZus{}tree}\PYG{p}{,} \PYG{n}{sos} \PYG{o}{=} \PYG{n}{mycen}\PYG{o}{.}\PYG{n}{compute\PYGZus{}centroid}\PYG{p}{(}\PYG{n}{TimeTreeSet}\PYG{p}{(}\PYG{p}{)}\PYG{p}{)}
\end{sphinxVerbatim}


\subsection{Centroid variations}
\label{\detokenize{summary:centroid-variations}}
\sphinxAtStartPar
The variation parameter of a \sphinxcode{\sphinxupquote{Centroid}} has to be one in {[}“inc\_sub”, “greedy”{]} (TODO: Still WIP).


\begin{savenotes}\sphinxattablestart
\centering
\begin{tabulary}{\linewidth}[t]{|T|T|}
\hline
\sphinxstyletheadfamily 
\sphinxAtStartPar
Variation
&\sphinxstyletheadfamily 
\sphinxAtStartPar
Description
\\
\hline
\sphinxAtStartPar
greedy
&
\sphinxAtStartPar
Computes a centroid via the greedy path and neighbourhood search. Only considering the tree with the most imporved SoS value in each iteration.
\\
\hline
\sphinxAtStartPar
inc\_sub
&
\sphinxAtStartPar
Starts with a subsample of trees from the set, computes the greedy centroid variant and adds more trees to the subsample until all trees are part of the sample.
\\
\hline
\sphinxAtStartPar
WIP
&
\sphinxAtStartPar
WIP
\\
\hline
\sphinxAtStartPar
WIP
&
\sphinxAtStartPar
WIP
\\
\hline
\sphinxAtStartPar
WIP
&
\sphinxAtStartPar
WIP
\\
\hline
\end{tabulary}
\par
\sphinxattableend\end{savenotes}


\subsection{Computing the SoS}
\label{\detokenize{summary:computing-the-sos}}

\subsection{Sampling the neighbourhood}
\label{\detokenize{summary:sampling-the-neighbourhood}}

\section{Frechet Mean}
\label{\detokenize{summary:frechet-mean}}

\chapter{Indices and tables}
\label{\detokenize{index:indices-and-tables}}\begin{itemize}
\item {} 
\sphinxAtStartPar
\DUrole{xref,std,std-ref}{genindex}

\item {} 
\sphinxAtStartPar
\DUrole{xref,std,std-ref}{modindex}

\item {} 
\sphinxAtStartPar
\DUrole{xref,std,std-ref}{search}

\end{itemize}



\renewcommand{\indexname}{Index}
\printindex
\end{document}
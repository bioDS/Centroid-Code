%% Generated by Sphinx.
\def\sphinxdocclass{report}
\documentclass[letterpaper,10pt,english]{sphinxmanual}
\ifdefined\pdfpxdimen
   \let\sphinxpxdimen\pdfpxdimen\else\newdimen\sphinxpxdimen
\fi \sphinxpxdimen=.75bp\relax
\ifdefined\pdfimageresolution
    \pdfimageresolution= \numexpr \dimexpr1in\relax/\sphinxpxdimen\relax
\fi
%% let collapsible pdf bookmarks panel have high depth per default
\PassOptionsToPackage{bookmarksdepth=5}{hyperref}

\PassOptionsToPackage{warn}{textcomp}
\usepackage[utf8]{inputenc}
\ifdefined\DeclareUnicodeCharacter
% support both utf8 and utf8x syntaxes
  \ifdefined\DeclareUnicodeCharacterAsOptional
    \def\sphinxDUC#1{\DeclareUnicodeCharacter{"#1}}
  \else
    \let\sphinxDUC\DeclareUnicodeCharacter
  \fi
  \sphinxDUC{00A0}{\nobreakspace}
  \sphinxDUC{2500}{\sphinxunichar{2500}}
  \sphinxDUC{2502}{\sphinxunichar{2502}}
  \sphinxDUC{2514}{\sphinxunichar{2514}}
  \sphinxDUC{251C}{\sphinxunichar{251C}}
  \sphinxDUC{2572}{\textbackslash}
\fi
\usepackage{cmap}
\usepackage[T1]{fontenc}
\usepackage{amsmath,amssymb,amstext}
\usepackage{babel}



\usepackage{tgtermes}
\usepackage{tgheros}
\renewcommand{\ttdefault}{txtt}



\usepackage[Bjarne]{fncychap}
\usepackage{sphinx}

\fvset{fontsize=auto}
\usepackage{geometry}


% Include hyperref last.
\usepackage{hyperref}
% Fix anchor placement for figures with captions.
\usepackage{hypcap}% it must be loaded after hyperref.
% Set up styles of URL: it should be placed after hyperref.
\urlstyle{same}

\addto\captionsenglish{\renewcommand{\contentsname}{Contents:}}

\usepackage{sphinxmessages}
\setcounter{tocdepth}{1}



\title{tetres}
\date{Aug 15, 2022}
\release{1.0.0}
\author{Lars Berling}
\newcommand{\sphinxlogo}{\vbox{}}
\renewcommand{\releasename}{Release}
\makeindex
\begin{document}

\pagestyle{empty}
\sphinxmaketitle
\pagestyle{plain}
\sphinxtableofcontents
\pagestyle{normal}
\phantomsection\label{\detokenize{index::doc}}



\chapter{Working with time trees}
\label{\detokenize{trees:working-with-time-trees}}\label{\detokenize{trees::doc}}
\begin{sphinxShadowBox}
\sphinxstyletopictitle{Contents}
\begin{itemize}
\item {} 
\sphinxAtStartPar
\phantomsection\label{\detokenize{trees:id3}}{\hyperref[\detokenize{trees:working-with-time-trees}]{\sphinxcrossref{Working with time trees}}}
\begin{itemize}
\item {} 
\sphinxAtStartPar
\phantomsection\label{\detokenize{trees:id4}}{\hyperref[\detokenize{trees:the-timetree-class}]{\sphinxcrossref{The TimeTree class}}}
\begin{itemize}
\item {} 
\sphinxAtStartPar
\phantomsection\label{\detokenize{trees:id5}}{\hyperref[\detokenize{trees:timetree-attributes}]{\sphinxcrossref{TimeTree attributes}}}

\item {} 
\sphinxAtStartPar
\phantomsection\label{\detokenize{trees:id6}}{\hyperref[\detokenize{trees:ete3-functionalities}]{\sphinxcrossref{ete3 functionalities}}}

\end{itemize}

\item {} 
\sphinxAtStartPar
\phantomsection\label{\detokenize{trees:id7}}{\hyperref[\detokenize{trees:the-timetreeset-class}]{\sphinxcrossref{The TimeTreeSet class}}}
\begin{itemize}
\item {} 
\sphinxAtStartPar
\phantomsection\label{\detokenize{trees:id8}}{\hyperref[\detokenize{trees:reading-trees}]{\sphinxcrossref{Reading Trees}}}

\end{itemize}

\item {} 
\sphinxAtStartPar
\phantomsection\label{\detokenize{trees:id9}}{\hyperref[\detokenize{trees:general-functions}]{\sphinxcrossref{General Functions}}}
\begin{itemize}
\item {} 
\sphinxAtStartPar
\phantomsection\label{\detokenize{trees:id10}}{\hyperref[\detokenize{trees:working-with-findpath-path-and-c-memory}]{\sphinxcrossref{Working with findpath\_path and c memory}}}

\end{itemize}

\item {} 
\sphinxAtStartPar
\phantomsection\label{\detokenize{trees:id11}}{\hyperref[\detokenize{trees:classes-for-the-c-library}]{\sphinxcrossref{Classes for the c library}}}
\begin{itemize}
\item {} 
\sphinxAtStartPar
\phantomsection\label{\detokenize{trees:id12}}{\hyperref[\detokenize{trees:node}]{\sphinxcrossref{NODE}}}

\item {} 
\sphinxAtStartPar
\phantomsection\label{\detokenize{trees:id13}}{\hyperref[\detokenize{trees:tree}]{\sphinxcrossref{TREE}}}

\item {} 
\sphinxAtStartPar
\phantomsection\label{\detokenize{trees:id14}}{\hyperref[\detokenize{trees:treelist}]{\sphinxcrossref{TREELIST}}}

\end{itemize}

\item {} 
\sphinxAtStartPar
\phantomsection\label{\detokenize{trees:id15}}{\hyperref[\detokenize{trees:class-converter-functions}]{\sphinxcrossref{Class converter functions}}}

\end{itemize}

\end{itemize}
\end{sphinxShadowBox}


\section{The TimeTree class}
\label{\detokenize{trees:the-timetree-class}}
\sphinxAtStartPar
A \sphinxcode{\sphinxupquote{TimeTree}} can be initialized with a given newick string using the  constructor and its format options.
Additionally, a \sphinxcode{\sphinxupquote{TREE}} object ({\hyperref[\detokenize{trees:c-classes}]{\sphinxcrossref{\DUrole{std,std-ref}{Classes for the c library}}}}) is generated and saved in the \sphinxcode{\sphinxupquote{TimeTree}} and used for efficient RNNI distance computations.


\subsection{TimeTree attributes}
\label{\detokenize{trees:timetree-attributes}}

\begin{savenotes}\sphinxattablestart
\centering
\begin{tabulary}{\linewidth}[t]{|T|T|}
\hline
\sphinxstyletheadfamily 
\sphinxAtStartPar
Method
&\sphinxstyletheadfamily 
\sphinxAtStartPar
Description
\\
\hline
\sphinxAtStartPar
\sphinxcode{\sphinxupquote{TimeTree.etree}}
&
\sphinxAtStartPar
returns the \sphinxcode{\sphinxupquote{ete3.Tree}} object
\\
\hline
\sphinxAtStartPar
\sphinxcode{\sphinxupquote{TimeTree.ctree}}
&
\sphinxAtStartPar
returns the respective \sphinxcode{\sphinxupquote{TREE}} object
\\
\hline
\sphinxAtStartPar
\sphinxcode{\sphinxupquote{len(TimeTree)}}
&
\sphinxAtStartPar
returns the number of leaves of the \sphinxcode{\sphinxupquote{TimeTree}}
\\
\hline
\sphinxAtStartPar
\sphinxcode{\sphinxupquote{TimeTree.fp\_distance(t)}}
&
\sphinxAtStartPar
returns the findpath distance to another \sphinxcode{\sphinxupquote{TimeTree}} \sphinxstyleemphasis{t}
\\
\hline
\sphinxAtStartPar
\sphinxcode{\sphinxupquote{TimeTree.fp\_path(t)}}
&
\sphinxAtStartPar
returns a \sphinxcode{\sphinxupquote{TREE\_LIST}} object, allocated memory needs to be freed!
\\
\hline
\sphinxAtStartPar
\sphinxcode{\sphinxupquote{TimeTree.get\_newick(format)}}
&
\sphinxAtStartPar
returns the \sphinxstyleemphasis{write()} function of the \sphinxcode{\sphinxupquote{ete3.Tree}} with the specified \sphinxstyleemphasis{format}, defaults to \sphinxstyleemphasis{format=5}
\\
\hline
\sphinxAtStartPar
\sphinxcode{\sphinxupquote{TimeTree.copy()}}
&
\sphinxAtStartPar
returns a deep copy of the current \sphinxcode{\sphinxupquote{TimeTree}}
\\
\hline
\sphinxAtStartPar
\sphinxcode{\sphinxupquote{TimeTree.neighbours()}}
&
\sphinxAtStartPar
returns a list of \sphinxcode{\sphinxupquote{TimeTree}}’s containing all neighbours at distance \sphinxstyleemphasis{1}
\\
\hline
\sphinxAtStartPar
\sphinxcode{\sphinxupquote{TimeTree.rank\_neighbours()}}
&
\sphinxAtStartPar
returns a list of \sphinxcode{\sphinxupquote{TimeTree}}’s containing only neighbours one \sphinxstyleemphasis{rank move} away
\\
\hline
\sphinxAtStartPar
\sphinxcode{\sphinxupquote{TimeTree.nni\_neighbours()}}
&
\sphinxAtStartPar
returns a list of \sphinxcode{\sphinxupquote{TimeTree}}’s containing only neighbours one \sphinxstyleemphasis{NNI move} away
\\
\hline
\sphinxAtStartPar
\sphinxcode{\sphinxupquote{TimeTree.nwk\_to\_cluster()}}
&
\sphinxAtStartPar
computes the set of all clades present in the given \sphinxcode{\sphinxupquote{TimeTree}}
\\
\hline
\sphinxAtStartPar
\sphinxcode{\sphinxupquote{TimeTree.apply\_new\_taxa\_map()}}
&
\sphinxAtStartPar
applies a new taxa map (in form of a dictionary) to a \sphinxcode{\sphinxupquote{TimeTree}}
\\
\hline
\end{tabulary}
\par
\sphinxattableend\end{savenotes}

\sphinxAtStartPar
This is an example of how to access the different attributes of a TimeTree object:

\begin{sphinxVerbatim}[commandchars=\\\{\}]
\PYG{k+kn}{from} \PYG{n+nn}{tetres}\PYG{n+nn}{.}\PYG{n+nn}{trees}\PYG{n+nn}{.}\PYG{n+nn}{time\PYGZus{}trees} \PYG{k+kn}{import} \PYG{n}{TimeTree}\PYG{p}{,} \PYG{n}{free\PYGZus{}tree\PYGZus{}list}


\PYG{c+c1}{\PYGZsh{} Initialize a time tree from a newick string}
\PYG{n}{tt} \PYG{o}{=} \PYG{n}{TimeTree}\PYG{p}{(}\PYG{l+s+s2}{\PYGZdq{}}\PYG{l+s+s2}{((1:3,5:3):1,(4:2,(3:1,2:1):1):2);}\PYG{l+s+s2}{\PYGZdq{}}\PYG{p}{)}

\PYG{n}{tt}\PYG{o}{.}\PYG{n}{ctree}  \PYG{c+c1}{\PYGZsh{} the TREE class object}

\PYG{n}{tt}\PYG{o}{.}\PYG{n}{etree}  \PYG{c+c1}{\PYGZsh{} the ete3.Tree object}

\PYG{n+nb}{len}\PYG{p}{(}\PYG{n}{tt}\PYG{p}{)}  \PYG{c+c1}{\PYGZsh{} Number of leaves in the tree tt \PYGZhy{}\PYGZhy{}\PYGZgt{} 5}

\PYG{n}{tt}\PYG{o}{.}\PYG{n}{fp\PYGZus{}distance}\PYG{p}{(}\PYG{n}{tt}\PYG{p}{)}  \PYG{c+c1}{\PYGZsh{} Distance to another TimeTree \PYGZhy{}\PYGZhy{}\PYGZgt{} 0}

\PYG{n}{path} \PYG{o}{=} \PYG{n}{tt}\PYG{o}{.}\PYG{n}{fp\PYGZus{}path}\PYG{p}{(}\PYG{n}{tt}\PYG{p}{)}  \PYG{c+c1}{\PYGZsh{} A shortest path to another TimeTree \PYGZhy{}\PYGZhy{}\PYGZgt{} []}
\PYG{n}{free\PYGZus{}tree\PYGZus{}list}\PYG{p}{(}\PYG{n}{path}\PYG{p}{)}  \PYG{c+c1}{\PYGZsh{} Allocated memory needs to be freed after usage}

\PYG{n}{tt}\PYG{o}{.}\PYG{n}{get\PYGZus{}newick}\PYG{p}{(}\PYG{p}{)}  \PYG{c+c1}{\PYGZsh{} Returns the newick string in ete3 format=5}

\PYG{n}{ttc} \PYG{o}{=} \PYG{n}{tt}\PYG{o}{.}\PYG{n}{copy}\PYG{p}{(}\PYG{p}{)}  \PYG{c+c1}{\PYGZsh{} ttc contains a deep copy of the TimeTree tt}

\PYG{n}{tt}\PYG{o}{.}\PYG{n}{neighbours}\PYG{p}{(}\PYG{p}{)}  \PYG{c+c1}{\PYGZsh{} a list of TimeTree objects each at distance one to tt}

\PYG{n}{tt}\PYG{o}{.}\PYG{n}{rank\PYGZus{}neighbours}\PYG{p}{(}\PYG{p}{)}  \PYG{c+c1}{\PYGZsh{} list of TimeTree obtained by doing all possible rank moves on tt}

\PYG{n}{tt}\PYG{o}{.}\PYG{n}{nni\PYGZus{}neighbours}\PYG{p}{(}\PYG{p}{)}  \PYG{c+c1}{\PYGZsh{} list of TimeTree obtained by doing all possible NNI moves on tt}

\PYG{n}{tt}\PYG{o}{.}\PYG{n}{nwk\PYGZus{}to\PYGZus{}cluster}\PYG{p}{(}\PYG{p}{)}  \PYG{c+c1}{\PYGZsh{} returns set of all clades in the tree}

\PYG{n}{tt}\PYG{o}{.}\PYG{n}{apply\PYGZus{}new\PYGZus{}taxa\PYGZus{}map}\PYG{p}{(}\PYG{n}{new\PYGZus{}map}\PYG{p}{,} \PYG{n}{old\PYGZus{}map}\PYG{p}{)}  \PYG{c+c1}{\PYGZsh{} Will apply the new taxa map to the tree}
\end{sphinxVerbatim}


\subsection{ete3 functionalities}
\label{\detokenize{trees:ete3-functionalities}}
\sphinxAtStartPar
Via the \sphinxcode{\sphinxupquote{ete3.Tree}} object the respective function of the \sphinxcode{\sphinxupquote{ete3}} package are available for a \sphinxcode{\sphinxupquote{TimeTree}} object.
For example drawing and saving a tree to a file:

\begin{sphinxVerbatim}[commandchars=\\\{\}]
\PYG{k+kn}{from} \PYG{n+nn}{tetres}\PYG{n+nn}{.}\PYG{n+nn}{trees}\PYG{n+nn}{.}\PYG{n+nn}{time\PYGZus{}trees} \PYG{k+kn}{import} \PYG{n}{TimeTree}

\PYG{n}{tt} \PYG{o}{=} \PYG{n}{TimeTree}\PYG{p}{(}\PYG{l+s+s2}{\PYGZdq{}}\PYG{l+s+s2}{((1:3,5:3):1,(4:2,(3:1,2:1):1):2);}\PYG{l+s+s2}{\PYGZdq{}}\PYG{p}{)}

\PYG{c+c1}{\PYGZsh{} Automatically save the tree to a specific file\PYGZus{}path location}
\PYG{n}{tt}\PYG{o}{.}\PYG{n}{etree}\PYG{o}{.}\PYG{n}{render}\PYG{p}{(}\PYG{l+s+s1}{\PYGZsq{}}\PYG{l+s+s1}{file\PYGZus{}path\PYGZus{}string}\PYG{l+s+s1}{\PYGZsq{}}\PYG{p}{)}

\PYG{c+c1}{\PYGZsh{} Defining a layout to display internal node names in the plot}
\PYG{k}{def} \PYG{n+nf}{my\PYGZus{}layout}\PYG{p}{(}\PYG{n}{node}\PYG{p}{)}\PYG{p}{:}
    \PYG{k}{if} \PYG{n}{node}\PYG{o}{.}\PYG{n}{is\PYGZus{}leaf}\PYG{p}{(}\PYG{p}{)}\PYG{p}{:}
        \PYG{c+c1}{\PYGZsh{} If terminal node, draws its name}
        \PYG{n}{name\PYGZus{}face} \PYG{o}{=} \PYG{n}{ete3}\PYG{o}{.}\PYG{n}{AttrFace}\PYG{p}{(}\PYG{l+s+s2}{\PYGZdq{}}\PYG{l+s+s2}{name}\PYG{l+s+s2}{\PYGZdq{}}\PYG{p}{)}
    \PYG{k}{else}\PYG{p}{:}
        \PYG{c+c1}{\PYGZsh{} If internal node, draws label with smaller font size}
        \PYG{n}{name\PYGZus{}face} \PYG{o}{=} \PYG{n}{ete3}\PYG{o}{.}\PYG{n}{AttrFace}\PYG{p}{(}\PYG{l+s+s2}{\PYGZdq{}}\PYG{l+s+s2}{name}\PYG{l+s+s2}{\PYGZdq{}}\PYG{p}{,} \PYG{n}{fsize}\PYG{o}{=}\PYG{l+m+mi}{10}\PYG{p}{)}
    \PYG{c+c1}{\PYGZsh{} Adds the name face to the image at the preferred position}
    \PYG{n}{ete3}\PYG{o}{.}\PYG{n}{faces}\PYG{o}{.}\PYG{n}{add\PYGZus{}face\PYGZus{}to\PYGZus{}node}\PYG{p}{(}\PYG{n}{name\PYGZus{}face}\PYG{p}{,} \PYG{n}{node}\PYG{p}{,} \PYG{n}{column}\PYG{o}{=}\PYG{l+m+mi}{0}\PYG{p}{,} \PYG{n}{position}\PYG{o}{=}\PYG{l+s+s2}{\PYGZdq{}}\PYG{l+s+s2}{branch\PYGZhy{}right}\PYG{l+s+s2}{\PYGZdq{}}\PYG{p}{)}

\PYG{n}{ts} \PYG{o}{=} \PYG{n}{ete3}\PYG{o}{.}\PYG{n}{TreeStyle}\PYG{p}{(}\PYG{p}{)}
\PYG{n}{ts}\PYG{o}{.}\PYG{n}{show\PYGZus{}leaf\PYGZus{}name} \PYG{o}{=} \PYG{k+kc}{False}
\PYG{n}{ts}\PYG{o}{.}\PYG{n}{layout\PYGZus{}fn} \PYG{o}{=} \PYG{n}{my\PYGZus{}layout}
\PYG{n}{ts}\PYG{o}{.}\PYG{n}{show\PYGZus{}branch\PYGZus{}length} \PYG{o}{=} \PYG{k+kc}{True}
\PYG{n}{ts}\PYG{o}{.}\PYG{n}{show\PYGZus{}scale} \PYG{o}{=} \PYG{k+kc}{False}

\PYG{c+c1}{\PYGZsh{} Will open a separate plot window, which also allows interactive changes and saving the image}
\PYG{n}{tt}\PYG{o}{.}\PYG{n}{etree}\PYG{o}{.}\PYG{n}{show}\PYG{p}{(}\PYG{n}{tree\PYGZus{}style}\PYG{o}{=}\PYG{n}{ts}\PYG{p}{)}
\end{sphinxVerbatim}

\sphinxAtStartPar
See the \sphinxcode{\sphinxupquote{ete3}} \sphinxhref{http://etetoolkit.org/docs/latest/tutorial/tutorial\_drawing.html}{documentation} for more options.


\section{The TimeTreeSet class}
\label{\detokenize{trees:the-timetreeset-class}}
\sphinxAtStartPar
A \sphinxcode{\sphinxupquote{TimeTreeSet}} is an iterable list of \sphinxcode{\sphinxupquote{TimeTree}} objects, which is initialized with a nexus file (as returned by a BEAST2 analysis), hence it contains a taxa map.


\begin{savenotes}\sphinxattablestart
\centering
\begin{tabulary}{\linewidth}[t]{|T|T|}
\hline
\sphinxstyletheadfamily 
\sphinxAtStartPar
Method
&\sphinxstyletheadfamily 
\sphinxAtStartPar
Description
\\
\hline
\sphinxAtStartPar
\sphinxcode{\sphinxupquote{TimeTreeSet.map}}
&
\sphinxAtStartPar
a dictionary conataining the taxa to integer translation from the nexus file
\\
\hline
\sphinxAtStartPar
\sphinxcode{\sphinxupquote{TimeTreeSet.trees}}
&
\sphinxAtStartPar
a list of \sphinxcode{\sphinxupquote{TimeTree}} objects
\\
\hline
\sphinxAtStartPar
\sphinxcode{\sphinxupquote{TimeTreeSet{[}i{]}}}
&
\sphinxAtStartPar
returns the \sphinxcode{\sphinxupquote{TimeTree}} at \sphinxcode{\sphinxupquote{TimeTreeSet.trees{[}i{]}}}
\\
\hline
\sphinxAtStartPar
\sphinxcode{\sphinxupquote{len(TimeTreeSet)}}
&
\sphinxAtStartPar
returns the number of trees in the list \sphinxcode{\sphinxupquote{TimeTreeSet.trees}}
\\
\hline
\sphinxAtStartPar
\sphinxcode{\sphinxupquote{TimeTreeSet.fp\_distance(i, j)}}
&
\sphinxAtStartPar
returns the distances between the trees at postition i and j
\\
\hline
\sphinxAtStartPar
\sphinxcode{\sphinxupquote{TimeTreeSet.fp\_path(i, j)}}
&
\sphinxAtStartPar
returns a shortest path (\sphinxcode{\sphinxupquote{TREE\_LIST}}) between the trees at postition i and j
\\
\hline
\sphinxAtStartPar
\sphinxcode{\sphinxupquote{TimeTreeSet.copy()}}
&
\sphinxAtStartPar
returns a copy of the list of :class:{\color{red}\bfseries{}\textasciigrave{}}TimeTree\textasciigrave{}s
\\
\hline
\sphinxAtStartPar
\sphinxcode{\sphinxupquote{TimeTreeSet.get\_common\_clades()}}
&
\sphinxAtStartPar
returns and computes the set of shared clades among all trees in the set
\\
\hline
\sphinxAtStartPar
\sphinxcode{\sphinxupquote{TimeTreeSet.change\_mapping(new\_map)}}
&
\sphinxAtStartPar
Will apply the given new taxa map to all trees in the set
\\
\hline
\end{tabulary}
\par
\sphinxattableend\end{savenotes}


\subsection{Reading Trees}
\label{\detokenize{trees:reading-trees}}
\sphinxAtStartPar
A TimeTreeSet object can be initialized with a path to a nexus file.

\begin{sphinxVerbatim}[commandchars=\\\{\}]
\PYG{k+kn}{from} \PYG{n+nn}{tetres}\PYG{n+nn}{.}\PYG{n+nn}{trees}\PYG{n+nn}{.}\PYG{n+nn}{time\PYGZus{}trees} \PYG{k+kn}{import} \PYG{n}{TimeTreeSet}\PYG{p}{,} \PYG{n}{free\PYGZus{}tree\PYGZus{}list}


\PYG{c+c1}{\PYGZsh{} Initializing with a path to a nexus tree file}
\PYG{n}{tts} \PYG{o}{=} \PYG{n}{TimeTreeSet}\PYG{p}{(}\PYG{l+s+s2}{\PYGZdq{}}\PYG{l+s+s2}{path\PYGZus{}to\PYGZus{}nexus\PYGZus{}file.nex}\PYG{l+s+s2}{\PYGZdq{}}\PYG{p}{)}

\PYG{n}{tts}\PYG{o}{.}\PYG{n}{map}  \PYG{c+c1}{\PYGZsh{} a dictionary keys:int and values:string(taxa)}

\PYG{n}{tts}\PYG{o}{.}\PYG{n}{trees}  \PYG{c+c1}{\PYGZsh{} A list of TimeTree objects}

\PYG{k}{for} \PYG{n}{tree} \PYG{o+ow}{in} \PYG{n}{tts}\PYG{p}{:}
    \PYG{c+c1}{\PYGZsh{} tree is a TimeTree object}
    \PYG{o}{.}\PYG{o}{.}\PYG{o}{.}
\PYG{n}{tts}\PYG{p}{[}\PYG{l+m+mi}{0}\PYG{p}{]}  \PYG{c+c1}{\PYGZsh{} trees are accessible via the index}

\PYG{n+nb}{len}\PYG{p}{(}\PYG{n}{tts}\PYG{p}{)}  \PYG{c+c1}{\PYGZsh{} Returns the number of trees in the TimeTreeSet object}

\PYG{n}{tts}\PYG{o}{.}\PYG{n}{fp\PYGZus{}distance}\PYG{p}{(}\PYG{n}{i}\PYG{p}{,} \PYG{n}{j}\PYG{p}{)}  \PYG{c+c1}{\PYGZsh{} Returns the distance between trees i and j}
\PYG{n}{path} \PYG{o}{=} \PYG{n}{tts}\PYG{o}{.}\PYG{n}{fp\PYGZus{}path}\PYG{p}{(}\PYG{n}{i}\PYG{p}{,} \PYG{n}{j}\PYG{p}{)}  \PYG{c+c1}{\PYGZsh{} Returns a shortest path between trees i and j}
\PYG{n}{free\PYGZus{}tree\PYGZus{}list}\PYG{p}{(}\PYG{n}{path}\PYG{p}{)}  \PYG{c+c1}{\PYGZsh{} Allocated memory needs to be freed after usage}
\end{sphinxVerbatim}


\section{General Functions}
\label{\detokenize{trees:general-functions}}
\sphinxAtStartPar
A list of the functions available from the module ‘tetres.trees.time\_trees’.


\begin{savenotes}\sphinxattablestart
\centering
\begin{tabulary}{\linewidth}[t]{|T|T|}
\hline
\sphinxstyletheadfamily 
\sphinxAtStartPar
Function
&\sphinxstyletheadfamily 
\sphinxAtStartPar
Description
\\
\hline
\sphinxAtStartPar
\sphinxcode{\sphinxupquote{time\_trees.neighbourhood(tree)}}
&
\sphinxAtStartPar
returns a list of \sphinxcode{\sphinxupquote{TimeTree}} objects containing the one\sphinxhyphen{}neighbours of tree
\\
\hline
\sphinxAtStartPar
\sphinxcode{\sphinxupquote{time\_trees.get\_rank\_neighbours(tree)}}
&
\sphinxAtStartPar
returns a list of \sphinxcode{\sphinxupquote{TimeTree}} objects containing the rank neighbours of tree
\\
\hline
\sphinxAtStartPar
\sphinxcode{\sphinxupquote{time\_trees.get\_nni\_neighbours(tree)}}
&
\sphinxAtStartPar
returns a list of \sphinxcode{\sphinxupquote{TimeTree}} objects containing the NNI neighbours of tree
\\
\hline
\sphinxAtStartPar
\sphinxcode{\sphinxupquote{time\_trees.read\_nexus(file)}}
&
\sphinxAtStartPar
returns a list of \sphinxcode{\sphinxupquote{TimeTree}} objects contained in given the nexus file
\\
\hline
\sphinxAtStartPar
\sphinxcode{\sphinxupquote{time\_trees.get\_mapping\_dict(file)}}
&
\sphinxAtStartPar
returns a \sphinxcode{\sphinxupquote{dictionary}} containing the taxa to integer translation of the given file
\\
\hline
\sphinxAtStartPar
\sphinxcode{\sphinxupquote{time\_trees.findpath\_distance(t1, t2)}}
&
\sphinxAtStartPar
Computes the distance between t1 and t2, returns \sphinxcode{\sphinxupquote{int}}
\\
\hline
\sphinxAtStartPar
\sphinxcode{\sphinxupquote{time\_trees.findpath\_path(t1, t2)}}
&
\sphinxAtStartPar
Computes the path between t1 and t2, returns \sphinxcode{\sphinxupquote{TREE\_LIST}}, after usage memory needs to be freed!
\\
\hline
\end{tabulary}
\par
\sphinxattableend\end{savenotes}

\begin{sphinxadmonition}{note}{Note:}
\sphinxAtStartPar
Both functions \sphinxcode{\sphinxupquote{time\_trees.findpath\_distance(t1, t2)}} and \sphinxcode{\sphinxupquote{time\_trees.findpath\_path(t1, t2)}}
can be called with t1 and t2 being either a \sphinxcode{\sphinxupquote{TREE}}, \sphinxcode{\sphinxupquote{TimeTree}} or \sphinxcode{\sphinxupquote{ete3.Tree}}, both have to be the same type!
\end{sphinxadmonition}

\begin{sphinxadmonition}{note}{Note:}
\sphinxAtStartPar
When using \sphinxcode{\sphinxupquote{time\_trees.findpath\_path(t1, t2)}} the c code is allocating memory to the returned object.
This memory needs to be freed with the \sphinxcode{\sphinxupquote{time\_trees.free\_tree\_list(tree\_list)}} function to avoid memory leaks, see more info below!
\end{sphinxadmonition}


\subsection{Working with findpath\_path and c memory}
\label{\detokenize{trees:working-with-findpath-path-and-c-memory}}
\sphinxAtStartPar
When using the \sphinxcode{\sphinxupquote{time\_trees.findpath\_path(t1, t2)}} implementation it is important to free the memory of the
returned \sphinxcode{\sphinxupquote{TREE\_LIST}} object. When calling the function the package will also throw a UserWarning indicating this.
Below are some examples of how to use the findpath\_path implementation and the underlying class \sphinxcode{\sphinxupquote{TREE\_LIST}}.

\begin{sphinxVerbatim}[commandchars=\\\{\}]
\PYG{k+kn}{from} \PYG{n+nn}{tetres}\PYG{n+nn}{.}\PYG{n+nn}{trees}\PYG{n+nn}{.}\PYG{n+nn}{time\PYGZus{}trees} \PYG{k+kn}{import} \PYG{n}{TimeTreeSet}\PYG{p}{,} \PYG{n}{free\PYGZus{}tree\PYGZus{}list}

\PYG{n}{t1} \PYG{o}{=} \PYG{n}{TimeTree}\PYG{p}{(}\PYG{p}{)}
\PYG{n}{t2} \PYG{o}{=} \PYG{n}{TimeTree}\PYG{p}{(}\PYG{p}{)}

\PYG{n}{path} \PYG{o}{=} \PYG{n}{findpath\PYGZus{}path}\PYG{p}{(}\PYG{n}{t1}\PYG{o}{.}\PYG{n}{ctree}\PYG{p}{,} \PYG{n}{t2}\PYG{o}{.}\PYG{n}{ctree}\PYG{p}{)}  \PYG{c+c1}{\PYGZsh{} Will throw a UserWarning}
\PYG{n}{free\PYGZus{}tree\PYGZus{}list}\PYG{p}{(}\PYG{n}{path}\PYG{p}{)}  \PYG{c+c1}{\PYGZsh{} Free the memory allocated by c}

\PYG{c+c1}{\PYGZsh{} Calling findpath\PYGZus{}path without the UserWarning being printed}
\PYG{k}{with} \PYG{n}{warnings}\PYG{o}{.}\PYG{n}{catch\PYGZus{}warnings}\PYG{p}{(}\PYG{p}{)}\PYG{p}{:}
    \PYG{c+c1}{\PYGZsh{} Ignores the \PYGZsq{}Free memory\PYGZsq{} warning issued by findpath\PYGZus{}path}
    \PYG{n}{warnings}\PYG{o}{.}\PYG{n}{simplefilter}\PYG{p}{(}\PYG{l+s+s2}{\PYGZdq{}}\PYG{l+s+s2}{ignore}\PYG{l+s+s2}{\PYGZdq{}}\PYG{p}{)}
    \PYG{c+c1}{\PYGZsh{} All following calls do the same thing, but the memory is not being freed}
    \PYG{n}{path} \PYG{o}{=} \PYG{n}{findpath\PYGZus{}path}\PYG{p}{(}\PYG{n}{t1}\PYG{p}{,} \PYG{n}{t2}\PYG{p}{)}
    \PYG{n}{path} \PYG{o}{=} \PYG{n}{findpath\PYGZus{}path}\PYG{p}{(}\PYG{n}{t1}\PYG{o}{.}\PYG{n}{ctree}\PYG{p}{,} \PYG{n}{t2}\PYG{o}{.}\PYG{n}{ctree}\PYG{p}{)}
    \PYG{n}{path} \PYG{o}{=} \PYG{n}{findpath\PYGZus{}path}\PYG{p}{(}\PYG{n}{t1}\PYG{o}{.}\PYG{n}{etree}\PYG{p}{,} \PYG{n}{t2}\PYG{o}{.}\PYG{n}{etree}\PYG{p}{)}

\PYG{c+c1}{\PYGZsh{} Use the c code to free the memory}
\PYG{k+kn}{from} \PYG{n+nn}{ctypes} \PYG{k+kn}{import} \PYG{n}{CDLL}
\PYG{k+kn}{from} \PYG{n+nn}{tetres}\PYG{n+nn}{.}\PYG{n+nn}{trees}\PYG{n+nn}{.}\PYG{n+nn}{\PYGZus{}ctrees} \PYG{k+kn}{import} \PYG{n}{TREE\PYGZus{}LIST}
\PYG{n}{lib} \PYG{o}{=} \PYG{n}{CDLL}\PYG{p}{(}\PYG{l+s+sa}{f}\PYG{l+s+s2}{\PYGZdq{}}\PYG{l+s+s2}{.../tetres/trees/findpath.so}\PYG{l+s+s2}{\PYGZdq{}}\PYG{p}{)}
\PYG{n}{lib}\PYG{o}{.}\PYG{n}{free\PYGZus{}treelist}\PYG{o}{.}\PYG{n}{argtypes} \PYG{o}{=} \PYG{p}{[}\PYG{n}{TREE\PYGZus{}LIST}\PYG{p}{]}
\PYG{n}{lib}\PYG{o}{.}\PYG{n}{free\PYGZus{}treelist}\PYG{p}{(}\PYG{n}{path}\PYG{p}{)}
\end{sphinxVerbatim}


\section{Classes for the c library}
\label{\detokenize{trees:classes-for-the-c-library}}\label{\detokenize{trees:c-classes}}
\sphinxAtStartPar
These classes are found in the \sphinxcode{\sphinxupquote{\_ctrees.py}} module.
The corresponding CDLL c library is generated from \sphinxcode{\sphinxupquote{findpath.c}}.


\subsection{NODE}
\label{\detokenize{trees:node}}\begin{itemize}
\item {} 
\sphinxAtStartPar
\sphinxcode{\sphinxupquote{parent}}: index of the parent node (int, defaults to \sphinxhyphen{}1)

\item {} 
\sphinxAtStartPar
\sphinxcode{\sphinxupquote{children{[}2{]}}}: index of the two children ({[}int{]}, defaults to {[}\sphinxhyphen{}1, \sphinxhyphen{}1{]})

\item {} 
\sphinxAtStartPar
\sphinxcode{\sphinxupquote{time}}: Time of the node (int, defaults to 0)

\end{itemize}

\begin{sphinxadmonition}{note}{Note:}
\sphinxAtStartPar
The attribute \sphinxcode{\sphinxupquote{time}} is currently not being used!
\end{sphinxadmonition}


\subsection{TREE}
\label{\detokenize{trees:tree}}\begin{itemize}
\item {} 
\sphinxAtStartPar
\sphinxcode{\sphinxupquote{num\_leaves}}: Number of leaves in the tree (int)

\item {} 
\sphinxAtStartPar
\sphinxcode{\sphinxupquote{tree}}: Points to a \sphinxcode{\sphinxupquote{NODE}} object (POINTER(\sphinxcode{\sphinxupquote{NODE}}))

\item {} 
\sphinxAtStartPar
\sphinxcode{\sphinxupquote{root\_time}}: Time of the root \sphinxcode{\sphinxupquote{Node}} (int)

\end{itemize}

\begin{sphinxadmonition}{note}{Note:}
\sphinxAtStartPar
The attribute \sphinxcode{\sphinxupquote{root\_time}} is currently not being used!
\end{sphinxadmonition}


\subsection{TREELIST}
\label{\detokenize{trees:treelist}}\begin{itemize}
\item {} 
\sphinxAtStartPar
\sphinxcode{\sphinxupquote{num\_trees}}: Number of trees in the list (int)

\item {} 
\sphinxAtStartPar
\sphinxcode{\sphinxupquote{trees}}: List of trees (POINTER(\sphinxcode{\sphinxupquote{TREE}}))

\end{itemize}


\section{Class converter functions}
\label{\detokenize{trees:class-converter-functions}}
\sphinxAtStartPar
These are found in \sphinxcode{\sphinxupquote{\_converter.py}} and convert one tree type into the other.
When converting a ctree to an ete3 Tree the branch lengths are discrete integers since the ctrees do not have a branch length annotation.


\begin{savenotes}\sphinxattablestart
\centering
\begin{tabulary}{\linewidth}[t]{|T|T|}
\hline
\sphinxstyletheadfamily 
\sphinxAtStartPar
Function
&\sphinxstyletheadfamily 
\sphinxAtStartPar
Description
\\
\hline
\sphinxAtStartPar
\sphinxcode{\sphinxupquote{\_converter.ete3\_to\_ctree(tree)}}
&
\sphinxAtStartPar
traverses an \sphinxcode{\sphinxupquote{ete3.Tree}} and construct the correct \sphinxcode{\sphinxupquote{TREE}}
\\
\hline
\sphinxAtStartPar
\sphinxcode{\sphinxupquote{\_converter.ctree\_to\_ete3(ctree)}}
&
\sphinxAtStartPar
recursively traverses a \sphinxcode{\sphinxupquote{TREE}} and generates an \sphinxcode{\sphinxupquote{ete3.Tree}}
\\
\hline
\end{tabulary}
\par
\sphinxattableend\end{savenotes}


\chapter{Summarizing trees}
\label{\detokenize{summary:summarizing-trees}}\label{\detokenize{summary::doc}}
\begin{sphinxShadowBox}
\sphinxstyletopictitle{Contents}
\begin{itemize}
\item {} 
\sphinxAtStartPar
\phantomsection\label{\detokenize{summary:id1}}{\hyperref[\detokenize{summary:summarizing-trees}]{\sphinxcrossref{Summarizing trees}}}
\begin{itemize}
\item {} 
\sphinxAtStartPar
\phantomsection\label{\detokenize{summary:id2}}{\hyperref[\detokenize{summary:the-centroid-class}]{\sphinxcrossref{The Centroid class}}}
\begin{itemize}
\item {} 
\sphinxAtStartPar
\phantomsection\label{\detokenize{summary:id3}}{\hyperref[\detokenize{summary:variation}]{\sphinxcrossref{Variation}}}
\begin{itemize}
\item {} 
\sphinxAtStartPar
\phantomsection\label{\detokenize{summary:id4}}{\hyperref[\detokenize{summary:greedy}]{\sphinxcrossref{Greedy}}}

\item {} 
\sphinxAtStartPar
\phantomsection\label{\detokenize{summary:id5}}{\hyperref[\detokenize{summary:inc-sub}]{\sphinxcrossref{Inc\_sub}}}

\item {} 
\sphinxAtStartPar
\phantomsection\label{\detokenize{summary:id6}}{\hyperref[\detokenize{summary:iter-sub}]{\sphinxcrossref{Iter\_sub}}}

\item {} 
\sphinxAtStartPar
\phantomsection\label{\detokenize{summary:id7}}{\hyperref[\detokenize{summary:separate}]{\sphinxcrossref{Separate}}}

\item {} 
\sphinxAtStartPar
\phantomsection\label{\detokenize{summary:id8}}{\hyperref[\detokenize{summary:onlyone}]{\sphinxcrossref{Onlyone}}}

\item {} 
\sphinxAtStartPar
\phantomsection\label{\detokenize{summary:id9}}{\hyperref[\detokenize{summary:update-with-one}]{\sphinxcrossref{update\sphinxhyphen{}with\sphinxhyphen{}one}}}

\item {} 
\sphinxAtStartPar
\phantomsection\label{\detokenize{summary:id10}}{\hyperref[\detokenize{summary:online}]{\sphinxcrossref{Online}}}

\end{itemize}

\item {} 
\sphinxAtStartPar
\phantomsection\label{\detokenize{summary:id11}}{\hyperref[\detokenize{summary:selecting-a-tree}]{\sphinxcrossref{Selecting a tree}}}

\item {} 
\sphinxAtStartPar
\phantomsection\label{\detokenize{summary:id12}}{\hyperref[\detokenize{summary:starting-tree}]{\sphinxcrossref{Starting tree}}}

\item {} 
\sphinxAtStartPar
\phantomsection\label{\detokenize{summary:id13}}{\hyperref[\detokenize{summary:subsample-size}]{\sphinxcrossref{Subsample size}}}

\item {} 
\sphinxAtStartPar
\phantomsection\label{\detokenize{summary:id14}}{\hyperref[\detokenize{summary:maximal-iterations}]{\sphinxcrossref{Maximal iterations}}}

\item {} 
\sphinxAtStartPar
\phantomsection\label{\detokenize{summary:id15}}{\hyperref[\detokenize{summary:computing-the-sos}]{\sphinxcrossref{Computing the SoS}}}

\item {} 
\sphinxAtStartPar
\phantomsection\label{\detokenize{summary:id16}}{\hyperref[\detokenize{summary:tree-logfile}]{\sphinxcrossref{Tree logfile}}}

\end{itemize}

\item {} 
\sphinxAtStartPar
\phantomsection\label{\detokenize{summary:id17}}{\hyperref[\detokenize{summary:annotation-of-a-centroid}]{\sphinxcrossref{Annotation of a centroid}}}

\item {} 
\sphinxAtStartPar
\phantomsection\label{\detokenize{summary:id18}}{\hyperref[\detokenize{summary:frechet-mean}]{\sphinxcrossref{Frechet Mean}}}

\end{itemize}

\end{itemize}
\end{sphinxShadowBox}


\section{The Centroid class}
\label{\detokenize{summary:the-centroid-class}}
\begin{sphinxVerbatim}[commandchars=\\\{\}]
\PYG{k}{class} \PYG{n+nc}{tetres}\PYG{o}{.}\PYG{n}{summary}\PYG{o}{.}\PYG{n}{Centroid}\PYG{p}{(}\PYG{n}{variation}\PYG{o}{=}\PYG{l+s+s2}{\PYGZdq{}}\PYG{l+s+s2}{greedy\PYGZus{}omp}\PYG{l+s+s2}{\PYGZdq{}}\PYG{p}{,} \PYG{n}{n\PYGZus{}cores}\PYG{o}{=}\PYG{k+kc}{None}\PYG{p}{,} \PYG{n}{select}\PYG{o}{=}\PYG{l+s+s1}{\PYGZsq{}}\PYG{l+s+s1}{random}\PYG{l+s+s1}{\PYGZsq{}}\PYG{p}{,} \PYG{n}{start}\PYG{o}{=}\PYG{l+s+s1}{\PYGZsq{}}\PYG{l+s+s1}{FM}\PYG{l+s+s1}{\PYGZsq{}}\PYG{p}{,} \PYG{n}{subsample\PYGZus{}size}\PYG{o}{=}\PYG{l+m+mi}{200}\PYG{p}{,}
             \PYG{n}{tree\PYGZus{}log\PYGZus{}file}\PYG{o}{=}\PYG{l+s+s2}{\PYGZdq{}}\PYG{l+s+s2}{\PYGZdq{}}\PYG{p}{,} \PYG{n}{max\PYGZus{}iterations}\PYG{o}{=}\PYG{k+kc}{None}\PYG{p}{)}
\end{sphinxVerbatim}

\sphinxAtStartPar
This is used to setup a Centroid object which then takes a \sphinxcode{\sphinxupquote{TimeTreeSet}} as input to compute the centroid summary tree.


\subsection{Variation}
\label{\detokenize{summary:variation}}
\sphinxAtStartPar
The variation parameter of a \sphinxcode{\sphinxupquote{Centroid}} has to be one in {[}“inc\_sub”, “greedy”{]} (TODO: Still WIP).


\begin{savenotes}\sphinxattablestart
\centering
\begin{tabulary}{\linewidth}[t]{|T|T|}
\hline
\sphinxstyletheadfamily 
\sphinxAtStartPar
Variation
&\sphinxstyletheadfamily 
\sphinxAtStartPar
Description
\\
\hline
\sphinxAtStartPar
{\hyperref[\detokenize{summary:var-greedy}]{\sphinxcrossref{\DUrole{std,std-ref}{Greedy}}}}
&
\sphinxAtStartPar
Computes a centroid via the greedy path and neighbourhood search. Only considering the tree with the most imporved SoS value in each iteration.
\\
\hline
\sphinxAtStartPar
{\hyperref[\detokenize{summary:var-incsub}]{\sphinxcrossref{\DUrole{std,std-ref}{Inc\_sub}}}}
&
\sphinxAtStartPar
Starts with a subsample of trees from the set, computes the greedy\_omp centroid variant and adds more trees to the subsample until all trees are part of the sample.
\\
\hline
\sphinxAtStartPar
{\hyperref[\detokenize{summary:var-itersub}]{\sphinxcrossref{\DUrole{std,std-ref}{Iter\_sub}}}}
&
\sphinxAtStartPar
Starts with a subsample of trees from the set, computes the greedy\_omp centroid variant and then resamples a new subset, using the previous centroid as the starting tree.
\\
\hline
\sphinxAtStartPar
{\hyperref[\detokenize{summary:var-separate}]{\sphinxcrossref{\DUrole{std,std-ref}{Separate}}}}
&
\sphinxAtStartPar
Only computes rank move neighbours if the tree contains all common clades of the tree set
\\
\hline
\sphinxAtStartPar
{\hyperref[\detokenize{summary:var-onlyone}]{\sphinxcrossref{\DUrole{std,std-ref}{Onlyone}}}}
&
\sphinxAtStartPar
Prefers either NNI or Rank moves and switches this if a local optimum is reached
\\
\hline
\sphinxAtStartPar
{\hyperref[\detokenize{summary:var-update-with-one}]{\sphinxcrossref{\DUrole{std,std-ref}{update\sphinxhyphen{}with\sphinxhyphen{}one}}}}
&
\sphinxAtStartPar
Similar to the incsub variation, only one tree at a time is added to the subsample
\\
\hline
\sphinxAtStartPar
{\hyperref[\detokenize{summary:var-online}]{\sphinxcrossref{\DUrole{std,std-ref}{Online}}}}
&
\sphinxAtStartPar
Mimicks an online approach where samples arrive one after another and the centroid is computed after each sample starting from the previous centroid
\\
\hline
\end{tabulary}
\par
\sphinxattableend\end{savenotes}


\subsubsection{Greedy}
\label{\detokenize{summary:greedy}}\label{\detokenize{summary:var-greedy}}
\begin{sphinxVerbatim}[commandchars=\\\{\}]
\PYG{n}{tetres}\PYG{o}{.}\PYG{n}{summary}\PYG{o}{.}\PYG{n}{Centroid}\PYG{p}{(}\PYG{n}{variation}\PYG{o}{=}\PYG{l+s+s2}{\PYGZdq{}}\PYG{l+s+s2}{greedy\PYGZus{}omp}\PYG{l+s+s2}{\PYGZdq{}}\PYG{p}{)}  \PYG{c+c1}{\PYGZsh{} default, using multiple processes in c!}
\PYG{n}{tetres}\PYG{o}{.}\PYG{n}{summary}\PYG{o}{.}\PYG{n}{Centroid}\PYG{p}{(}\PYG{n}{variation}\PYG{o}{=}\PYG{l+s+s2}{\PYGZdq{}}\PYG{l+s+s2}{greedy}\PYG{l+s+s2}{\PYGZdq{}}\PYG{p}{)}  \PYG{c+c1}{\PYGZsh{} pure python version}
\end{sphinxVerbatim}


\subsubsection{Inc\_sub}
\label{\detokenize{summary:inc-sub}}\label{\detokenize{summary:var-incsub}}
\sphinxAtStartPar
The parameter subsample\_size defines the size of the subsample of trees that is added each iteration.
The parameter max\_iterations defines the number of iterations, if it is None the regular break is defined whenever an
iteration is not successful at improving the previous centroid.
If it is an integer then it defines the number of iterations that will subsample, if it is 0 the start tree will be returned.

\begin{sphinxVerbatim}[commandchars=\\\{\}]
\PYG{n}{tetres}\PYG{o}{.}\PYG{n}{summary}\PYG{o}{.}\PYG{n}{Centroid}\PYG{p}{(}\PYG{n}{variation}\PYG{o}{=}\PYG{l+s+s2}{\PYGZdq{}}\PYG{l+s+s2}{inc\PYGZus{}sub}\PYG{l+s+s2}{\PYGZdq{}}\PYG{p}{,} \PYG{n}{subsample\PYGZus{}size}\PYG{o}{=}\PYG{l+m+mi}{500}\PYG{p}{,} \PYG{n}{max\PYGZus{}iterations}\PYG{o}{=}\PYG{k+kc}{None}\PYG{p}{)}
\end{sphinxVerbatim}


\subsubsection{Iter\_sub}
\label{\detokenize{summary:iter-sub}}\label{\detokenize{summary:var-itersub}}
\sphinxAtStartPar
The parameter subsample\_size defines the size of the subsample of trees that is sampled each iteration.
The parameter max\_iterations defines the number of iterations, if it is None the regular break is defined whenever an
iteration is not successful at improving the previous centroid.
If it is an integer then it defines the number of iterations that will subsample, if it is 0 the start tree will be returned.

\begin{sphinxVerbatim}[commandchars=\\\{\}]
\PYG{n}{tetres}\PYG{o}{.}\PYG{n}{summary}\PYG{o}{.}\PYG{n}{Centroid}\PYG{p}{(}\PYG{n}{variation}\PYG{o}{=}\PYG{l+s+s2}{\PYGZdq{}}\PYG{l+s+s2}{iter\PYGZus{}sub}\PYG{l+s+s2}{\PYGZdq{}}\PYG{p}{,} \PYG{n}{subsample\PYGZus{}size}\PYG{o}{=}\PYG{l+m+mi}{500}\PYG{p}{,} \PYG{n}{max\PYGZus{}iterations}\PYG{o}{=}\PYG{k+kc}{None}\PYG{p}{)}
\end{sphinxVerbatim}


\subsubsection{Separate}
\label{\detokenize{summary:separate}}\label{\detokenize{summary:var-separate}}
\sphinxAtStartPar
Will only use one move, current implementation is for NNI moves only, needs to be switched in source code (\_variations.py, line 147).

\begin{sphinxVerbatim}[commandchars=\\\{\}]
\PYG{n}{tetres}\PYG{o}{.}\PYG{n}{summary}\PYG{o}{.}\PYG{n}{Centroid}\PYG{p}{(}\PYG{n}{variation}\PYG{o}{=}\PYG{l+s+s2}{\PYGZdq{}}\PYG{l+s+s2}{separate}\PYG{l+s+s2}{\PYGZdq{}}\PYG{p}{)}
\end{sphinxVerbatim}


\subsubsection{Onlyone}
\label{\detokenize{summary:onlyone}}\label{\detokenize{summary:var-onlyone}}
\sphinxAtStartPar
Will always do one move (starting with rank moves as of current implementation) and switch the move type whenever a local optimum is found.

\begin{sphinxVerbatim}[commandchars=\\\{\}]
\PYG{n}{tetres}\PYG{o}{.}\PYG{n}{summary}\PYG{o}{.}\PYG{n}{Centroid}\PYG{p}{(}\PYG{n}{variation}\PYG{o}{=}\PYG{l+s+s2}{\PYGZdq{}}\PYG{l+s+s2}{onlyone}\PYG{l+s+s2}{\PYGZdq{}}\PYG{p}{)}
\end{sphinxVerbatim}


\subsubsection{update\sphinxhyphen{}with\sphinxhyphen{}one}
\label{\detokenize{summary:update-with-one}}\label{\detokenize{summary:var-update-with-one}}
\sphinxAtStartPar
Similar to the inc\sphinxhyphen{}sub variation but only one new tree is added in each iteration.

\begin{sphinxVerbatim}[commandchars=\\\{\}]
\PYG{n}{tetres}\PYG{o}{.}\PYG{n}{summary}\PYG{o}{.}\PYG{n}{Centroid}\PYG{p}{(}\PYG{n}{variation}\PYG{o}{=}\PYG{l+s+s2}{\PYGZdq{}}\PYG{l+s+s2}{update\PYGZus{}with\PYGZus{}one}\PYG{l+s+s2}{\PYGZdq{}}\PYG{p}{)}
\end{sphinxVerbatim}


\subsubsection{Online}
\label{\detokenize{summary:online}}\label{\detokenize{summary:var-online}}
\sphinxAtStartPar
Mimicks an online approach where the trees arrive one by one in the given order.

\begin{sphinxVerbatim}[commandchars=\\\{\}]
\PYG{n}{tetres}\PYG{o}{.}\PYG{n}{summary}\PYG{o}{.}\PYG{n}{Centroid}\PYG{p}{(}\PYG{n}{variation}\PYG{o}{=}\PYG{l+s+s2}{\PYGZdq{}}\PYG{l+s+s2}{online}\PYG{l+s+s2}{\PYGZdq{}}\PYG{p}{)}
\end{sphinxVerbatim}


\subsection{Selecting a tree}
\label{\detokenize{summary:selecting-a-tree}}
\sphinxAtStartPar
This is only the case if multiple trees have the same SoS value.
The defualt is random and the options are either random, first or last.
The second two options are depending on the ordering which is dictated by the way the neighbourhood
is computed.


\subsection{Starting tree}
\label{\detokenize{summary:starting-tree}}
\sphinxAtStartPar
There are the options to start with the last, the first or any given index of tree from the given tree set.
The default option however is the sorted Frechet Mean tree (ref), see the doc on FM for more detail.


\subsection{Subsample size}
\label{\detokenize{summary:subsample-size}}
\sphinxAtStartPar
This is used by some variations and can be set to any integer number (default is 200).
This number indicates the size of the subsample that the variation will use in its iterations.
See the incsub or itersub variations


\subsection{Maximal iterations}
\label{\detokenize{summary:maximal-iterations}}
\sphinxAtStartPar
This is used to limit the number of iterations the iterative subsampling and increasing subsampling centroid versions are computing.
If it is None (default) then the regular break points of those variations apply, otherwise it will only compute upto max\_iteration
many iterations.


\subsection{Computing the SoS}
\label{\detokenize{summary:computing-the-sos}}
\sphinxAtStartPar
The n\_cores parameters defines the number of cores to use, if \sphinxhyphen{}1 all available cores are used (default).


\subsection{Tree logfile}
\label{\detokenize{summary:tree-logfile}}
\sphinxAtStartPar
This option will write the trees of each centroid iteration to the given file path.
This includes the actual centroid as the last tree.
Can be used for further analysis.

\sphinxAtStartPar
Note that for incsub for example the tree is logged after an iteration on the subsample.
This results in much smaller log files.


\section{Annotation of a centroid}
\label{\detokenize{summary:annotation-of-a-centroid}}
\sphinxAtStartPar
To keep the discrete ranks of a centroid use this annotation method.
Each rank get assigned the average height of that rank in the given tree set, guaranteed to keep the same ranked tree after the annotation.

\begin{sphinxVerbatim}[commandchars=\\\{\}]
\PYG{k+kn}{from} \PYG{n+nn}{tetres}\PYG{n+nn}{.}\PYG{n+nn}{summary}\PYG{n+nn}{.}\PYG{n+nn}{annotate\PYGZus{}centroid} \PYG{k+kn}{import} \PYG{n}{annotate\PYGZus{}centroid}
\PYG{n}{cen}\PYG{p}{,} \PYG{n}{sos} \PYG{o}{=} \PYG{n}{Centroid}\PYG{p}{(}\PYG{n}{variation}\PYG{o}{=}\PYG{l+s+s2}{\PYGZdq{}}\PYG{l+s+s2}{greedy}\PYG{l+s+s2}{\PYGZdq{}}\PYG{p}{)}\PYG{o}{.}\PYG{n}{compute\PYGZus{}centroid}\PYG{p}{(}\PYG{n}{my\PYGZus{}tts}\PYG{p}{)}  \PYG{c+c1}{\PYGZsh{} centroid of the TimeTreeSet my\PYGZus{}tts}
\PYG{n}{annotated\PYGZus{}cen} \PYG{o}{=} \PYG{n}{annotate\PYGZus{}centroid}\PYG{p}{(}\PYG{n}{cen}\PYG{p}{,} \PYG{n}{my\PYGZus{}tts}\PYG{p}{)}  \PYG{c+c1}{\PYGZsh{} Annotation with the branch lengths from the TimeTreeSet my\PYGZus{}tts}
\PYG{c+c1}{\PYGZsh{} the annotated\PYGZus{}cen is a TimeTree object for further use such as writing the newick to a file}
\end{sphinxVerbatim}


\section{Frechet Mean}
\label{\detokenize{summary:frechet-mean}}
\sphinxAtStartPar
A version of Sturms algorithm adapted to the RNNI tree space, based on computing shortest paths with the findpath algorithm.

\begin{sphinxVerbatim}[commandchars=\\\{\}]
\PYG{k+kn}{from} \PYG{n+nn}{tetres}\PYG{n+nn}{.}\PYG{n+nn}{summary}\PYG{n+nn}{.}\PYG{n+nn}{frechet\PYGZus{}mean} \PYG{k+kn}{import} \PYG{n}{frechet\PYGZus{}mean}\PYG{p}{,} \PYG{n}{frechet\PYGZus{}mean\PYGZus{}sort}
\PYG{n}{fm} \PYG{o}{=} \PYG{n}{frechet\PYGZus{}mean}\PYG{p}{(}\PYG{n}{my\PYGZus{}tts}\PYG{p}{)}  \PYG{c+c1}{\PYGZsh{} random selection of trees}
\PYG{n}{fm\PYGZus{}sort} \PYG{o}{=} \PYG{n}{frechet\PYGZus{}mean\PYGZus{}sort}\PYG{p}{(}\PYG{n}{my\PYGZus{}tts}\PYG{p}{)}  \PYG{c+c1}{\PYGZsh{} trees are sorted from highest to lowest Sum of squared distances}
\PYG{c+c1}{\PYGZsh{} The idea is that tres with low sum of squared distances are used in the end of the algorithm to refine the tree}
\end{sphinxVerbatim}


\chapter{Indices and tables}
\label{\detokenize{index:indices-and-tables}}\begin{itemize}
\item {} 
\sphinxAtStartPar
\DUrole{xref,std,std-ref}{genindex}

\item {} 
\sphinxAtStartPar
\DUrole{xref,std,std-ref}{modindex}

\item {} 
\sphinxAtStartPar
\DUrole{xref,std,std-ref}{search}

\end{itemize}



\renewcommand{\indexname}{Index}
\printindex
\end{document}